\documentclass[oneside]{homework} %%Change `twoside' to `oneside' if you are printing only on the one side of each sheet.
\usepackage{amsmath}
\usepackage{amssymb,mathtools}
\usepackage{graphicx}
\graphicspath{}

\studname{Alex Wong}
\studmail{asw2181@columbia.edu}
\coursename{COMS E6232}
\hwNo{3}
\uni{asw2181}

\begin{document}
\maketitle
\skipevenpage

\problemNo{1}
{\large a.} \textbf{Lemma 1a} There is always a directed cut that contains at least $1/4$ of the edges.
\newline

\textbf{Proof:} For $i = 1,...,|N|$ we assign node $i$ to set $S$ with probability $1/2$ and to set $N-S$ with probability $1/2$. For each directed edge $(i, j)$, let $X_{(i,j)} = 1$ if $(i, j) \in$ directed cut from S to N-S, else 0 otherwise. Thus, we get the equation: $$E[X_{(i,j)}] = P[(i,j) \in \text{directed cut from S to N-S}] = P[i \in S, j \in N-S] = 1/4$$ \hfill\qed
\newline
\newline
{\large b.} 
\newline
\newline
{\large c.} The worst-case approximation ratio of the simple local search algorithm is unbounded. Take for example a node $i$ and some number n of nodes where all edges are directed towards $i$. If we put node $i$ in set $S$ and the rest of the nodes in set $N-S$, then all the edges are directed from $N-S$ to $S$, giving a cut with 0 edges since only edges from $S$ to $N-S$ count. Using only one local move will not improve the cut, regardless of which node we decide to move. If we move node $i$ to $N-S$, then all nodes will be on that side. If we move any of the $n$ nodes from $N-S$ to $S$, then the edge will reside within $S$, still not improving the cut. We can see that the maximum cut would have the $n$ number of nodes in set $S$ and node $i$ in set $N-S$, which would give us a cut with $n$ number of edges. Thus, we can see that the worst-case approximation ratio is unbounded using the simple local search algorithm since we can scale $n$ to any arbitrarily large number.
\newline
\newline
{\large d.} We can improve the local search so that it guarantees ratio 1/4. The improvement of the local search algorithm is summarized below:
\newline
\newline
Given some directed graph $D = (N,E)$ and a random assignment of the nodes $i = 1,...,|N|$ into either set $S$ or $N-S$, while $\exists$ a node $i$ that has more neighbors on the same side than the opposite side, move $i$ to the opposite side. Based on slides 23 and 24 of the Lecture 5 slides, we will end up with $\geq 1/2|E|$ of edges that go between the sets $S$ and $N-S$. From this, we find the value of the directed cut of these edges. Then, we move all the current nodes that are in $S$ to $N-S$ and the current nodes of $N-S$ to $S$ and find the value of the directed cut of these edges. We take the configuration of $S$ and $N-S$ based on the larger of the two directed cuts that we found, which at least one of them $\geq 1/4|E|$. One of the cuts will be $\geq 1/4|E|$ because there are a total of $\geq 1/2|E|$ edges in the cut, so if $\leq 1/4|E|$ edges go from $S$ to $N-S$ in one configuration, then there must be $\geq 1/4|E|$ edges going from $N-S$ to $S$, so we would move the nodes to orient the direction of the edges to go from $S$ to $N-S$.

\problemNo{2}
{\large a.} The quadratic program for this problem is given below:
$$\text{max}\frac{1}{2}\sum\limits_{i<j}(1 +(y_i \cdot y_j \cdot z_{ij}))$$
$$\text{s.t.} \ y_i \in \{-1, +1\}, \forall i = 1,...,n$$
$$z_{ij} \in \{-1, +1\}, i < j$$
The correspondence between the problem and the quadratic program are as follows:
$$\text{object} \ i \in \text{class} \ S \Leftrightarrow y_i = +1$$
$$\text{object} \ i \in \text{class} \ S' \Leftrightarrow y_i = -1$$
$$\text{similar constraint} \ i \sim j \Leftrightarrow z_{ij} = +1$$
$$\text{dissimilar constraint} \ i \nsim j \Leftrightarrow z_{ij} = -1$$ 
\newline
For objects $i, j$ that are in the same class with a similarity constraint: $$y_i y_j z_{ij} = (-1)(-1)z_{ij} \ \text{or} \ (+1)(+1)z_{ij} = 1 \cdot z_{ij} = (1)(1) = 1 \Leftrightarrow 1 + y_i y_j z_{ij} = 2$$
For objects $i, j$ that are in the same class with a dissimilarity constraint: $$y_i y_j z_{ij} = (-1)(-1)z_{ij} \ \text{or} \ (+1)(+1)z_{ij} = 1 \cdot z_{ij} = (1)(-1) = -1 \Leftrightarrow 1 + y_i y_j z_{ij} = 0$$
For objects $i, j$ that are in different classes with a similarity constraint: $$y_i y_j z_{ij} = (-1)(+1)z_{ij} \ \text{or} \ (+1)(-1)z_{ij} = -1 \cdot z_{ij} = (-1)(1) = -1 \Leftrightarrow 1 + y_i y_j z_{ij} = 0$$
For objects $i, j$ that are in different classes with a dissimilarity constraint: $$y_i y_j z_{ij} = (-1)(+1)z_{ij} \ \text{or} \ (+1)(-1)z_{ij} = -1 \cdot z_{ij} = (-1)(-1) = 1 \Leftrightarrow 1 + y_i y_j z_{ij} = 2$$
where the objective function is maximizing the number of satisfied constraints.

\problemNo{3}
{\large a.} \textbf{Lemma 3a} If $I$ is an independent set of $G$ then $I^k$ is an independent set of $G^{(k)}$.
\newline

\textbf{Proof:} For $I$ to be an independent set of $G$, there must be no adjacent edges between any of the nodes of $I$. Given the definition of $E^{(k)}$ of $G^{(k)}$, $$E^{(k)} = \{(u, v) \in N^{(k)} \times N^{(k)} \mid \exists i, j \in [k], (u_i, u_j) \in E \text{ or } (v_i, v_j) \in E \text{ or } (u_i, v_j) \in E\}$$ we can see that there are no edges between any pairs $(u, v)$ of $k$-tuples of nodes that only contain nodes of $I$. When taking the Cartesian product of $I$, $I^k$, we create $k$-tuples using only those nodes from $I$ so $I^k \subseteq N^{(k)}$ since $I \subseteq N$. Also, as explained earlier, the $k$-tuples of $I^k$ do not have adjacent edges between them since they only contain nodes of $I$, thus showing that $I^k$ is an independent set of $G^{(k)}$. \hfill\qed
\newline
\newline
{\large b.} \textbf{Lemma 3b} If $J$ is an independent set of $G^{(k)}$ then we can construct in polynomial time an independent set $I$ of $G$ of size at least $|J|^{1/k}$ and conclude that $\alpha(G^{(k)}) = (\alpha(G))^k$
\newline

\textbf{Proof:} From Lemma 3a, if $I$ is an independent set of $G$, then $I^k$ is an independent set of $G^{(k)}$. Since we have defined $J$ to be an independent set of $G^{(k)}$, we can use Lemma 3a to say that $J = I^k$. The size of $J$ is then equal to the size of $I^k$, which is $|I|^k$, thus $|J| = |I|^k \Longrightarrow |I| = |J|^{1/k}$, showing that we can construct in polynomial time an independent set $I$ of $G$ of size at least $|J|^{1/k}$. Also, if given the maximum independent set of $G$, $\alpha(G)$, we can use our earlier conclusion and see that $(\alpha(G^{(k)}))^{1/k} = \alpha(G) \Longrightarrow \alpha(G^{(k)}) = (\alpha(G))^k$. \hfill\qed
\newline
\newline
{\large c.} \textbf{Lemma 3c} If the Maximum Independent Set problem can be approximated in polynomial time within some constant factor $c > 1$, then it has a PTAS.
\newline

\textbf{Proof:} We first define $\alpha(G)$ as the maximum independent set of $G$. From the PCP theorem, we can trivially say that the Maximum Independent Set problem has a 2-approximation algorithm. Let us define $I$ as an independent set of G that agrees with the 2-approximation algorithm. This means that the size of $I$ will be $1/2$ the size of the maximum independent set of $G$, thus $\frac{|\alpha(G)|}{|I|} = 2$. Now, using Lemma 3b, if we take the $k$-th power of the graph G, we know that $\alpha(G^{(k)}) = (\alpha(G))^k$ and by Lemma 3a, $I^k$ is an independent set of $G^{(k)}$. Thus, for some $k$-th power graph $G^{(k)}$, the approximation ratio is $\frac{|\alpha(G)|^k}{|I|^k} = (\frac{|\alpha(G)|}{|I|})^k = 2^k$ which shows that the approximation ratio grows as we apply the approximation algorithm to larger and larger $k$-th powers of graph $G$. Thus, the Maximum Independent Set problem can not be approximated in polynomial time within some constant factor, which means it does not have a PTAS unless \textbf{P = NP}. \hfill\qed

\problemNo{4}
{\large a.} \textbf{Lemma 4a} MDAS can be trivially approximated within a factor of 2.
\newline

\textbf{Proof:} Suppose we have an arbitrary ordering of nodes $v_1, ..., v_n$ and we have two subsets of edges $A_1$ and $A_2$ where $$A_1 = \{(v_i, v_j) \mid (v_i, v_j) \in A, i < j\}$$ $$A_2 = \{(v_i, v_j) \mid (v_i, v_j) \in A, i > j\}$$ By separating the edges into these two subsets, the only way for $A_1$ to contain a cycle is if there is an edge where $i > j$, which would be contained in the $A_2$ subset, and the only way for $A_2$ to contain a cycle is if there is an edge where $ i < j$, which would be contained in the $A_1$ subset. Thus, neither subset will contain a cycle and at least one of the two subsets will contain at least $|A|/2$ edges since $A = A_1 + A_2$. Thus, MDAS can be trivially approximated within a factor of 2 by taking the larger of the two subsets. \hfill\qed
\newline
\newline
{\large b.} \textbf{Theorem 4.2} The Maximum Directed Acyclic Subgraph (MDAS) problem does not have a PTAS unless \textbf{P = NP}.
\newline

\textbf{Proof:} A known problem that does not have a PTAS is the Maximum Independent Set (MIS) problem. MIS does not have a PTAS for any graph with a maximum degree $\geq 3$. We can do a linear reduction $MIS(3) \leq_L MDAS$: Given an undirected graph $G = (N, E)$ with maximum degree 3, we construct a directed graph $D = (V, A)$ where $V = \{(u_1, u_2) \mid u \in N\}$ and $A = \{(u_1, u_2) \mid u \in N\}  \cup \{(u_2, v_1),(v_2, u_1) \mid (u, v) \in E\}$. We let $\alpha(G)$ denote the size of the maximum independent set of $G$ and $\gamma(D)$ denote the number of edges of the maximum acyclic subgraph of $D$. From this, we have the following lemma:
\newline

\textbf{Lemma 4.2.1} For all $\epsilon > 0$, if we are given an acyclic subgraph of $D$ that has at least $(1-(\epsilon/13))\gamma(D)$ edges, then we can compute in polynomial time an independent set $G$ that has at least $(1-\epsilon)\alpha(G)$ nodes.
\newline

\textbf{Proof}:
\newline
\textbf{(4.2a)} If $I$ is an independent set of $G$, then $D' = (V, A')$ where $A' = \{(u_1, u_2) \mid u \in I\} \cup \{(u_2, v_1), (v_2, u_1) \mid (u, v) \in E\}$ is an acyclic subgraph of $D$. We can see that all edges $\{(u_2, v_1), (v_2, u_1) \mid (u, v) \in E\}$ do not create a cycle because as we have shown in Lemma 4a, a set of edges $\{(v_i, v_j) \mid (v_i, v_j) \in A, i > j\}$ do not create a cycle. We also know that the independent set $I$ contains nodes that do not have any adjacent edges with each other. Thus, since $A'$ contains all the linear reduction of edges of $E$, the only way to keep the edge set acyclic would be to only add the edges $\{(u_1, u_2) \mid u \in N\}$ where the nodes do not have any adjacent edges, which would be the independent set $I$ and gives us $\{(u_1, u_2) \mid u \in I\}$. Thus, if $I$ is an independent set of $G$, then $A'$ is an acyclic subgraph of $D$.
\newline
\newline
\textbf{(4.2b)} Now we show that if $H$ is an acyclic subgraph of $D$ with $h$ edges, we can then derive efficiently from $H$ an independent set of $G$ with at least $h-2|E|$ nodes. As we have shown in \textbf{4.2a}, acyclic subgraphs of $D$ includes all edges $\{(u_2, v_1), (v_2, u_1) \mid (u, v) \in E\}$.  We can see that each edge of $E$ corresponds to two edges in $A$. If we took away all those edges from $H$, we would be left with edges $\{(u_1, u_2) \mid u \in N\}$ which we know are nodes that could create an independent set, as we have shown in \textbf{4.2a}. Thus, we can derive efficiently from $H$ an independent set of $G$ with at least $h-2|E|$ nodes.
\newline
\newline
\textbf{(4.2c)} Lastly, since $G$ has maximum degree 3, we can show that $\alpha(G) \geq |E|/6$. Without loss of generality, lets take a node $u \in N$ that has degree 3, which would imply 4 connected nodes. If we were to make all these nodes connected together, we would have a complete graph where each of the 4 nodes would have a maximum degree of 3. This complete graph has a 6 total edges. From this complete graph, we can only choose any one node as the maximum independent set $\alpha(G)$ as all nodes are adjacent to all other nodes. Thus, we can see that $\alpha(G) \geq |E|/6$.
\newline
\newline
Let us do the linear reduction from MIS to MDAS. From \textbf{4.2b} we know we can get an acyclic subgraph of $D$ with $|I| + 2|E|$ edges. If $I = \alpha(G)$, then getting an acyclic subgraph $D$ with $\alpha(G) + 2|E|$ would imply a maximum acyclic subgraph $\gamma(D)$. We can manipulate our conclusion from \textbf{4.2c} where $\alpha(G) \geq |E|/6 \Longrightarrow |E| \leq 6\cdot\alpha(G)$. Thus, we can say $$\gamma(D) = \alpha(G) + 2|E| \leq \alpha(G) + 12\cdot\alpha(G) = 13\cdot\alpha(G)$$ Based on this L-reduction, it satisfies the first property that $OPT_{MDAS} \leq \alpha \cdot OPT_{MIS}$ where $\alpha = 13$. We can also see it satisfies the second property $|C_1 - OPT_{MIS}| = \beta\cdot|C_2 - OPT_{MDAS}|$ where $\beta = 1$. Combining both properties, we get the relative error: $$\frac{|C_1 - OPT_{MIS}|}{OPT_{MIS}} \leq \frac{\alpha\cdot\beta\cdot|C_2 - OPT_{MDAS}|}{OPT_{MDAS}} \leq \alpha\beta\epsilon$$ Showing that the relative error of MIS is $13\cdot \epsilon$ in relation to the relative error of MDAS being $\epsilon$. Thus, if we wanted to express the relative error of MIS as just $\epsilon$, that would mean that the relative error of MDAS would be $\epsilon/13$. Thus for all $\epsilon > 0$, if we are given an acyclic subgraph of $D$ that has at least $(1-(\epsilon/13))\gamma(D)$ edges, then we can compute in polynomial time an independent set $G$ that has at least $(1-\epsilon)\alpha(G)$ nodes, which proves the lemma. \hfill\qed
\newline
\newline
But because MIS does not have a PTAS as we have proven in problem 3, and since $MIS \leq_L MDAS$, that tells us that MDAS also does not have a PTAS unless \textbf{P = NP}, thus the theorem is proved. \hfill\qed


\end{document}