\documentclass[oneside]{homework} %%Change `twoside' to `oneside' if you are printing only on the one side of each sheet.

\studname{Alex Wong}
\studmail{asw2181@columbia.edu}
\coursename{COMS E6232}
\hwNo{2}
\uni{asw2181}

\begin{document}
\maketitle
\skipevenpage

\problemNo{1}
{\large a.} Consider an instance where we have two items x and y where $s_x = \epsilon$, $v_x = 2\epsilon$, $s_y = B$, $v_y =B$, and $\epsilon << 1$. The $v_i/s_i$ ratio of item x is 2 and item y is 1. Thus, the Greedy algorithm will always pick item x regardless of how large B is and regardless how small $\epsilon$ is. So as B gets larger and/or $\epsilon$ gets smaller, the approximation ratio of the algorithm will always increase, thus showing that the approximation ratio of Greedy is not bounded by any constant. \hfill\qed
\newline
\newline
{\large b.} \textbf{Theorem 1b} The Modified Greedy algorithm achieves approximation ratio 2.
\newline

\textbf{Proof:} We first assume that the items are ordered in a non-increasing fashion according to the ratio $v_i/s_i$. Let's call the first item that does not fit in the knapsack using the Greedy algorithm as item $m$. We know that item $m$'s $v_i/s_i$ ratio $\geq$ items $m+1,...,n$'s $v_i/s_i$ ratio. Thus, if we are able to fit some fraction of item $m$ so that it fills up to the capacity of the knapsack, that solution will be $\geq$ OPT. We can define the fraction of item $m$ that fits into the knapsack as $\alpha$ where $\alpha = (B - \sum\limits_{i=1}^{m-1} v_i) / s_m$. Thus, $OPT \leq (\sum\limits_{i=1}^{m-1} v_i) + \alpha v_m$. Since $\alpha$ is some fraction $\leq 1$, we can also say that: $$OPT \leq (\sum\limits_{i=1}^{m-1} v_i) + \alpha v_m \leq (\sum\limits_{i=1}^{m-1} v_i) + v_m$$ From the inequality above, $\sum\limits_{i=1}^{m-1} v_i$ or $v_m$ must be at least $OPT/2$, showing that the Modified Greedy algorithm will always get a solution at least $OPT/2$, thus achieving an approximation ratio 2. \hfill\qed

\problemNo{2}
{\large1.} \textbf{Lemma 2.1} Let $T$ be the minimum spanning tree of the weighted complete graph $G$. Then cost($T$) $\leq$ OPT, where OPT is the cost of the optimal \textit{s-t} path.
\newline

\textbf{Proof:} We can prove this by contradiction. Suppose that cost($T$) $>$ OPT. We know that OPT must visit every node to create a path. Thus, if OPT $<$ cost($T$), it violates the very definition of a minimum spanning tree, proving that cost($T$) $>$ OPT can not be true. Thus, cost($T$) $\leq$ OPT. \hfill$\qed$
\newline
\newline
{\large2.} \textbf{Lemma 2.2} Let $U$ be the subset of nodes consisting of the nodes of $\{1,...,n\}$-$\{s,t\}$ that have odd degree in $T$ and the nodes in $\{s,t\}$ that have even degree in $T$. Then $U$ has an even number of nodes.
\newline

\textbf{Proof:} Suppose we partition $U$ into 2 subsets: $U_{odd}$ and $U_{even}$ where they contain nodes that have odd degree and nodes that have even degree, respectively. We know that in the minimum spanning tree $T$, there is an even number of nodes that have odd degree since $\sum\limits_{v\in T_{odd}}degree(v) = 2\cdot \# edges$. Knowing this, if $|U_{odd}|$ is odd, that means either $s$ or $t$ must have an odd degree, meaning the other must have an even degree and would be contained in $U_{even}$, so $|U_{even}|$ = 1.  Thus, $$|U| = |U_{odd}|+| U_{even}| = odd\# + 1 = even\# $$ which shows that $|U|$ is even. 

If $|U_{odd}|$ is even, then $s$ and $t$ must both have odd degrees or both have even degrees; if both are odd, then $|U_{even}|$ = 0 which trivially shows $|U|$ is even. If both are even, then $|U_{even}|$ = 2 and $$|U| = |U_{odd}|+| U_{even}| = even\# + 2 = even\# $$ which shows that $|U|$ is even. It must be noted that if $|U_{even}| = 2$ there must be other nodes in set $U$; $U$ can never just contain $s$ and $t$. Thus, the set U has an even number of nodes. \hfill\qed
\newline
\newline
{\large3.} \textbf{Lemma 2.3} Let $G[U]$ be the subgraph of $G$ induced by the subset $U$, M be a minimum-cost perfect matching in $G[U]$, and $P^*$ be a minimum-cost path from $s$ to $t$ that visits every node exactly once. Consider the nodes of $U$ in the order that they appear on the path $P^*$ and let $u_i$ be the $i$-th node of $U$ in this ordering: $ i=1,...,|U|$. Color red the edges of $P^*$ on the subpath from $u_1$ to $u_2$,..., from $u_{2k-1}$ to $u_{2k}$, where $|U|=2k$. Color blue the other edges of $P^*$. Then, cost(red edges) $\geq$ cost($M$).
\newline

\textbf{Proof:} We can prove this by contradiction. Suppose cost(red edges) $<$ cost($M$). We know that red edges are essentially a perfect matching on set $U$. If cost(red edges) $<$ cost($M$), then $M$ is not a minimum-cost perfect matching on set $U$, proving that cost(red edges) $<$ cost($M$) can not be true. Thus, cost(red edges) $\geq$ cost($M$). \hfill\qed
\newline
\newline
{\large4.} \textbf{Lemma 2.4} cost($T$) + cost(blue edges) $\geq$ 2 $\cdot$ cost($M$).
\newline

\textbf{Proof:} We first see that $T$ $\cup$ (blue edges of $P^*$) creates a connected Eulerian multigraph. We know this to be true because every vertex has an even degree. A Eulerian multigraph is essentially a multigraph that contains an Eulerian tour. Within a Eulerian tour, we know that we can create two disjoint perfect matchings of $U$ and the sum of their costs $\leq$ cost of the Eulerian tour due to triangle inequality. We also know that both of the disjoint perfect matchings can not have a cost $<$ cost($M$) or else $M$ can't be a minimum-cost perfect matching on $U$. Thus, $$2 \cdot cost(M) \leq cost(2\ disjoint\ perfect\ matchings) \leq cost(T) + cost(blue\ edges)$$ $$cost(T) + cost(blue\ edges) \geq 2 \cdot cost(M)$$ \hfill\qed
\newline
\newline
{\large5.} \textbf{Theorem 2.5} The variant of Christofides' algorithm achieves an approximation factor of 5/3 for the \textit{s-t path metric TSP} problem.
\newline

\textbf{Proof:} Combining our conclusions from Lemma 2.3 and 2.4, we can say that: 
\begin{equation} \label{eq:1} 
cost(T) + cost(blue\ edges) + cost(red\ edges) \geq 3 \cdot cost(M)
\end{equation} 
We also know that $cost(blue\ edges) + cost(red\ edges) = P^* = OPT$ and from Lemma 2.1 that $cost(T) \leq OPT$. Thus, we can modify equation 1 to be: $$cost(T) + OPT \geq 3 \cdot cost(M)$$ $$2 \cdot OPT \geq 3 \cdot cost(M)$$
\begin{equation} \label{eq:2} 
cost(M) \leq 2/3\ OPT
\end{equation}
Adding cost($T$) to both sides of equation 2 and using Lemma 2.1 that $cost(T) \leq OPT$, we finally get: $$cost(T) + cost(M) \leq cost(T) + 2/3\ OPT$$ $$cost(T) + cost(M) \leq OPT + 2/3\ OPT$$ $$cost(T) + cost(M) \leq 5/3\ OPT$$
Thus showing that the cost of the path computed by the variant of Christofides' algorithm is at most 5/3 times the cost of the optimal path. \hfill\qed
\newpage
{\large6.}
\end{document}