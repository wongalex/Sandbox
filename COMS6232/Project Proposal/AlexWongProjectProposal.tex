\documentclass[oneside]{homework} %%Change `twoside' to `oneside' if you are printing only on the one side of each sheet.
\usepackage{amsmath}
\usepackage{amssymb,mathtools}
\usepackage{graphicx}
\graphicspath{}

\studname{Alex Wong}
\studmail{asw2181@columbia.edu}
\coursename{COMS E6232}
\hwNo{4}
\uni{asw2181}

\begin{document}
\maketitle
\skipevenpage

\problemNo{1}
In the second lecture of this class when we were discussing TSP heuristics, I was fascinated by the idea of utilizing a disjoint cycle cover in order to derive an approximation ratio for Asymmetric TSP (ATSP). During the lecture, we only covered a logn approximation algorithm for ATSP while only briefly mentioning that there was a better approximation algorithm than logn approximation. I was curious on what this better algorithm entailed. Now, since I have the opportunity to research a topic of my choice, I want to specifically research more in depth of the Asymmetric TSP. The TSP and all its variants have been thought provoking for me but unfortunately, I would not be able to cover all those variants within this research paper due to the length constraints for this paper. My goal of this research paper is to provide a summary of the current state of the art P-time algorithms and their accompanying approximation ratios for ATSP and to provide an educated opinion on what I perceive will be the future direction of the research towards better ATSP approximation algorithms. I will also summarize on why we can not approximate within 75/74 in polynomial time for ATSP unless P = NP. The main papers that I will focus on will include ``An O(logn/loglogn)-approximation algorithm for the asymmetric traveling salesman problem" by Asadpoor etal, SODA 2010, and ``The elusive inapproximability of the TSP" by Lampis, SIGACT 2014. Also, I will supplement research on additional papers that I feel would complement my analysis and summarization of the main focus papers.
\end{document}